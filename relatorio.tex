% Compilado com "latexmk -pdf" e "latexmk -C" para limpar, do pacote "latexmk".

\documentclass[12pt,a4paper]{article}

\usepackage[T1]{fontenc}
\usepackage[utf8]{inputenc}
\usepackage[brazilian,brazil]{babel}
\usepackage[labelfont=bf]{caption}
\usepackage{indentfirst}
\linespread{1.5}
\setlength{\parindent}{1.3cm}
\setlength{\parskip}{0.2cm}

\begin{document}

\pagestyle{plain}

\begin{titlepage}
\begin{center}
	\textbf{{\large Universidade Federal do ABC -- UFABC}}

	\textbf{\large Sistemas Distribuídos}
	\vfill
	\textit{\Large Projeto Final - [nome]}
	\vfill
	\begin{flushright}
		\textbf{\large Eduardo Pinhata}

		\textbf{\large Ricardo Liang}

		\textbf{\large Vanessa Morita}
	\end{flushright}
	\vfill
\end{center}
\end{titlepage}

\tableofcontents

\pagebreak \section{Introdução}

Este projeto tem como objetivo desenvolver um jogo 2D \textit{multiplayer} de
tiros automatizável (\textit{scriptable}) por cada jogador. O protocolo de
comunicação é construído em cima do \textit{HTTP}, buscando-se seguir
princípios \textit{RESTful} no seu desenvolvimento.

\subsection{Mecânicas do Jogo}

Cada objeto do jogo, incluindo jogadores, são compostos por um conjunto de
atributos.

\subsubsection{\texttt{name}}

Nome do objeto.

\subsubsection{\texttt{password}}

Senha do objeto. Deve-se requisitá-la a um cliente que deseje fazer
alterações nos atributos.

\subsubsection{\texttt{type}}

Tipo do objeto. Pode ser \texttt{player} ou \texttt{projectile}.

\subsubsection{\texttt{hp}}

Vida de um objeto. Se chega a zero, o objeto é removido do jogo.

\subsubsection{\texttt{posx}, \texttt{posy}}

Representa a posição do objeto no espaço 2D. É desejável que dois objetos não
ocupem o mesmo espaço, e um servidor deve tratar tentativas de tais
ocorrências. Se um objeto \texttt{projectile} coincidir em posição com outro
objeto, deve reduzir seu \texttt{hp} e ser removido do jogo em seguida.

\subsubsection{\texttt{movx}, \texttt{movy}}

Representa a direção de movimento do jogador no espaço 2D. Um servidor deve, em
princípio, somar \texttt{movx} a \texttt{posx} e \texttt{movy} a \texttt{posy}
para movimentar um objeto.

\subsubsection{\texttt{lookx}, \texttt{looky}}

A direção para a qual o objeto está ``olhando''. É usado também para determinar
a direção na qual tiros feitos por jogadores irão se mover. Também pode ser
usado para implementar movimento lateral, neste caso penalizando a mobilidade.

\pagebreak \section{Servidor}

Uma \textit{API RESTful} representa recursos em um servidor através de
\textit{URL}s. De forma geral, o método \texttt{GET} obtém os dados de um
recurso, o método \texttt{POST} cria um recurso, e o método \texttt{PUT} altera
um recurso.

Cada objeto do jogo é um recurso, com sua \textit{URL} única. Estão contidos no
recurso \texttt{/game}, sendo da forma \texttt{/game/[string aleatória]}. Cada
cliente deve-se manter atualizado fazendo requisições repetidamente a
\texttt{/game} pela lista de \textit{URL}s de objetos, e também, à \textit{URL}
de cada objeto.

Para evitar sobrecarga na rede, recursos acessados pelo método GET bloqueiam a
requisição se o cabeçalho tiver o campo \texttt{If-Modified-Since},
especificando uma data igual ou superior ao \texttt{timestamp} de modificação
do recurso.

\subsection{Entradas e Saídas}

Uma entrada é o corpo do pacote de requisição e uma saída é o corpo do pacote
de resposta. Todas entradas e saídas são em formato JSON, podendo ser objetos
(encapsulados por ``\texttt{\{\}}'') ou vetores (encapsulados por
``\texttt{[]}'') de variados atributos dependendo do recurso e do método.

\subsection{Recursos e Métodos}

\subsubsection{\texttt{POST /game}}

Cria um recurso \texttt{[token]} no servidor correspondente a um jogador. É
acessível através de \texttt{/game/[token]}, que é retornado no campo
\texttt{Location} do cabeçalho HTTP de retorno. Requer uma senha para que o
criador faça modificações depois.

\begin{itemize}
	\item Entrada: Objeto
		\begin{itemize}
			\item \texttt{name}: string, opcional
			\item \texttt{password}: string
		\end{itemize}
	\item Código de retorno: \texttt{201 Created}
	\item Saída: Nenhum
\end{itemize}

\subsubsection{\texttt{GET /game/[token]}}

Obtém os atributos correspondentes a um jogador.

\begin{itemize}
	\item Entrada: Nenhum
	\item Código de retorno: \texttt{200 OK}
	\item Saída: Objeto
		\begin{itemize}
			\item \texttt{name}: string
			\item \texttt{type}: string
			\item \texttt{hp}: int
			\item \texttt{kills}: int
			\item \texttt{posx}: int
			\item \texttt{posy}: int
			\item \texttt{movx}: int
			\item \texttt{movy}: int
			\item \texttt{lookx}: int
			\item \texttt{looky}: int
		\end{itemize}
\end{itemize}

\subsubsection{\texttt{PUT /game/[token]}}

Atualiza, do jogador, atributos passados como entrada. Requer a senha usada
para criação.

\begin{itemize}
	\item Entrada: Objeto
		\begin{itemize}
			\item \texttt{password}: string
			\item \texttt{movement}: Objeto, opcional
			\item \texttt{movx}: int, opcional
			\item \texttt{movy}: int, opcional
			\item \texttt{lookx}: int, opcional
			\item \texttt{looky}: int, opcional
		\end{itemize}
	\item Código de retorno: \texttt{202 Accepted}
	\item Saída: Nenhum
\end{itemize}

\subsubsection{\texttt{POST /game/[token]}}

Cria um recurso em \texttt{/game} que representa um projétil atirado pelo
jogador.

\begin{itemize}
	\item Entrada: Objeto
		\begin{itemize}
			\item \texttt{password}: string
		\end{itemize}
	\item Código de retorno: \texttt{201 Created}
	\item Saída: Nenhum
\end{itemize}

\subsubsection{\texttt{GET /game}}

Obtém todos os \texttt{[token]}s representando cada jogador presente.

\begin{itemize}
	\item Entrada: Nenhum
	\item Código de retorno: \texttt{200 OK}
	\item Saída: Vetor: string
\end{itemize}

\pagebreak \section{Cliente}

\pagebreak \section{Resultados}

\pagebreak \section{Conclusão}

%\bibliographystyle{}
%\bibliography{}

\end{document}
