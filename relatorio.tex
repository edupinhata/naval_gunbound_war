% Compilado com "latexmk -pdf" e "latexmk -C" para limpar, do pacote "latexmk".

\documentclass[12pt,a4paper]{article}

\usepackage[T1]{fontenc}
\usepackage[utf8]{inputenc}
\usepackage[brazilian,brazil]{babel}
\usepackage[labelfont=bf]{caption}
\usepackage{indentfirst}
\linespread{1.3}

\begin{document}

\pagestyle{plain}

\begin{titlepage}
\begin{center}
	\textbf{{\large Universidade Federal do ABC}}

	\vfill

	\textbf{\large Sistemas Distribuídos - 2013-3}

	\textbf{\large Prof. Dr. Francisco Isidro Massetto}

	\textit{\Large Projeto Final - ``ScriptBattle''}

	\vfill

	\begin{flushright}
		\large Eduardo Pinhata

		\large Ricardo Liang

		\large Vanessa Morita
	\end{flushright}

	\vfill
\end{center}
\end{titlepage}

\tableofcontents

\pagebreak \section{Introdução}

Este projeto tem como objetivo desenvolver um jogo 2D \textit{multiplayer} de
tiros automatizável (\textit{scriptable}) por cada jogador. O protocolo de
comunicação é construído em cima do HTTP e o formato de transmissão de dados,
JSON. A linguagem usada para desenvolvimento foi Python por possuir classes
para construção de servidores e clientes HTTP e para manipulação de JSON na sua
biblioteca padrão.

O jogo é composto de objetos com variados atributos. A cada intervalo de tempo,
os atributos de um objeto são atualizados pelas mecânicas do jogo. Jogadores
são criados a partir de um \textit{script}, que é executado após tal
atualização, passando-se como parâmetros o estado dos demais jogadores, podendo
então alterar o estado do jogador mais uma vez.

\subsection{Mecânicas do Jogo}

Cada objeto do jogo, incluindo jogadores, são compostos por um conjunto de
atributos, que definem como seus estados serão alterados durante uma
atualização.  Objetos podem ser jogadores, projéteis ou pedras. O
\textit{script} de um jogador pode alterar seu movimento e criar um número
limitado de projéteis.  Estes, apenas navegam em uma direção até colidirem com
outro objeto, reduzindo sua vida. Também possuem uma distância máxima para não
se moverem indefinidamente.

\subsubsection{\texttt{hp}}

Vida de um objeto. Se chega a zero, o objeto é removido do jogo.

\subsubsection{\texttt{type}}

Tipo do objeto para diferenciar certos comportamentos e para que clientes
saibam como desenhá-los. Pode ser \texttt{player}, \texttt{projectile} ou
\texttt{rock}.

\subsubsection{\texttt{posx}, \texttt{posy}}

Representa a posição do objeto no espaço 2D. É desejável que dois objetos não
ocupem o mesmo espaço, e um servidor deve tratar tentativas de tais ocorrências
(colisões), parando o movimento do objeto. Adicionalmente, se um objeto
\texttt{projectile} colidir com outro, deve reduzir seu \texttt{hp} e ser
removido do jogo em seguida.

\subsubsection{\texttt{movx}, \texttt{movy}}

Representa a direção de movimento do jogador no espaço 2D. Um servidor deve, em
princípio, somar \texttt{movx} a \texttt{posx} e \texttt{movy} a \texttt{posy}
para movimentar um objeto. Tem valor zero para objetos do tipo \texttt{rock}.

\subsubsection{\texttt{lookx}, \texttt{looky}}

A direção para a qual o objeto está ``olhando''. É usado também para determinar
a direção na qual tiros feitos por jogadores irão se mover. Também pode ser
usado para diferenciar movimento frontal de lateral, beneficiando a mobilidade
do primeiro.

\subsubsection{\texttt{shots}}

A quantidade de objetos \texttt{projectile} presentes no jogo, criados por um
jogador. É incrementado ao atirar, e decrementado quando o tiro for removido do
jogo.

\subsubsection{\texttt{shooting}}

Booleano usado pelo servidor após a atualização dos atributos para determinar
se um novo projétil deve ser criado naquele momento.

\subsubsection{\texttt{kills}}

A quantidade de objetos cujo \texttt{hp} foram reduzidos a zero através de um
projétil de um jogador.

\pagebreak \section{Protocolo}

Por conta de usar-se o protocolo \textit{HTTP}, busca-se seguir princípios
\textit{RESTful} no desenvolvimento do protocolo. Assim, cada objeto do jogo
corresponde a um \textit{recurso} no servidor identificado por uma URI, por
exemplo, \texttt{/game/abc123}, onde \texttt{abc123} é o nome do recurso (URN)
e \texttt{http://localhost:8000/game/abc123} é a localização do recurso (URL).
Clientes devem realizar requisições, por exemplo, \texttt{GET} para obter as
informações sobre o recurso, \texttt{POST} para criar um novo, \texttt{PUT}
para alterá-lo.

Cada objeto do jogo tem uma URN única e está contido no recurso \texttt{/game},
sendo da forma \texttt{/game/[urn]}. Cada cliente deve-se manter atualizado
fazendo requisições repetidamente a \texttt{/game} pela lista de URNs de
objetos, e também, a cada objeto individualmente pelo seu estado.

Requisitar repetidamente o estado de um objeto, sem que ele tenha atualizações,
caracteriza espera ocupada. Assim, para evitar sobrecarga na rede, recursos
acessados pelo método \texttt{GET} bloqueiam a requisição se o cabeçalho conter
o campo \texttt{If-Modified-Since} padronizado pelo HTTP, especificando uma
data igual ou superior ao \textit{timestamp} de modificação do recurso.

\subsection{Entradas e Saídas}

Uma entrada é o corpo do pacote de requisição e uma saída é o corpo do pacote
de resposta. Todas entradas e saídas são em formato JSON, podendo ser objetos
(encapsulados por ``\texttt{\{\}}'') ou vetores (encapsulados por
``\texttt{[]}'') de variados atributos dependendo do recurso e do método.

\subsection{API HTTP}

\subsubsection{\texttt{GET /game}}

Obtém todos os URNs dos objetos presentes no jogo.

\begin{itemize}
	\item Entrada: Nenhum
	\item Código de retorno: \texttt{200 OK}
	\item Saída: Vetor JSON: \texttt{string}
\end{itemize}

Exemplo:

\begin{verbatim}
[
    "player1",
    "player2",
    "projectile1",
    "rock1",
    "rock2",
    "rock3"
]
\end{verbatim}

\subsubsection{\texttt{POST /game}}

Recebe um nome único, uma senha para modificações, e o script do jogador, e
cria um recurso de URI \texttt{/game/[nome]} no servidor. A URN é retornada no
campo \texttt{Location} do cabeçalho HTTP de retorno.

\begin{itemize}
	\item Entrada: Objeto JSON
		\begin{itemize}
			\item \texttt{name}: \texttt{string}
			\item \texttt{password}: \texttt{password}
			\item \texttt{script}: \texttt{string}
		\end{itemize}
	\item Código de retorno: \texttt{201 Created}
	\item Saída: Nenhum
\end{itemize}

Exemplo:

\begin{verbatim}
{
    "name": "player1",
    "password": "123456",
    "script": "import random
               attributes['movx'] = random.randrange(-1, 2)"
}
\end{verbatim}

\subsubsection{\texttt{GET /game/[nome]}}

Obtém os atributos correspondentes a um objeto.

\begin{itemize}
	\item Entrada: Nenhum
	\item Código de retorno: \texttt{200 OK}
	\item Saída: Objeto JSON
		\begin{itemize}
			\item \texttt{hp}: \texttt{int}
			\item \texttt{type}: \texttt{string}
			\item \texttt{posx}: \texttt{int}
			\item \texttt{posy}: \texttt{int}
			\item \texttt{movx}: \texttt{int}
			\item \texttt{movy}: \texttt{int}
			\item \texttt{lookx}: \texttt{int}
			\item \texttt{looky}: \texttt{int}
			\item \texttt{shots}: \texttt{int}
			\item \texttt{shooting}: \texttt{boolean}
			\item \texttt{kills}: \texttt{int}
		\end{itemize}
\end{itemize}

Exemplo:

\begin{verbatim}
{
    "hp": 5,
    "type": "rock",
    "posx": 5,
    "posy": 10,
    "movx": 0,
    "movy": 0,
    "lookx": -1,
    "looky": 1,
    "shots": 0,
    "shooting": false,
    "kills": 0
}
\end{verbatim}

\subsubsection{\texttt{PUT /game/[nome]}}

Recebe a senha para modificações e um script para substituir o antigo.

\begin{itemize}
	\item Entrada: Objeto JSON
		\begin{itemize}
			\item \texttt{password}: \texttt{string}
			\item \texttt{script}: \texttt{string}
		\end{itemize}
	\item Código de retorno: \texttt{202 Accepted}
	\item Saída: Nenhum
\end{itemize}

Exemplo:

\begin{verbatim}
{
    "password": "123456",
    "script": "import random
               attributes['movy'] = random.randrange(-1, 2)"
}
\end{verbatim}

\pagebreak \section{Resultados}

Preveniu-se a espera ocupada na escuta de atualizações no servidor pelo
cliente, porém notou-se que alguns objetos falham em receber atualizações
quando deveriam, isto porque existe um tempo entre o recebimento de uma
resposta do servidor e a criação de uma nova requisição para escuta, por mais
que o processamento da resposta seja feito em paralelo. Isto se tornou evidente
quando há muitos objetos no servidor, onde a probabilidade de não-recebimento
aumenta.

\pagebreak \section{Conclusão}

%\bibliographystyle{}
%\bibliography{}

\end{document}
